%%Modelo criado por Régis (c) 2011
%% http://latexbr.blogspot.com
%% twitter: @rg3915
\documentclass{sciposter}
\usepackage{preposter} %pacote com o preambulo.
%\usepackage[final]{graphics}
\usepackage{wrapfig,times}
%\usepackage[ansinew]{inputenc}
%\usepackage{amsmath}
%\usepackage{amsthm,amsfonts}
%\usepackage{amssymb,latexsym}
%\usepackage{natbib}
\usepackage{slashbox}
\usepackage{multicol}
\usepackage{multirow}
\usepackage{booktabs}
\usepackage{bm}

\definecolor{mainCol}{rgb}{1,1,1}
%\definecolor{BoxCol}{rgb}{0.13,0.67,0.29} % verde
\definecolor{BoxCol}{rgb}{0.105,0.390,0.605} % azul
\definecolor{TextCol}{rgb}{1,1,1}
\definecolor{SectionCol}{rgb}{0,0,0}

\usepackage[T1]{fontenc}
%\usepackage{Times}

\newcommand{\Z}{\mathbb Z} % conjunto dos números inteiros
\newcommand{\F}{\mathbb F} % sigma-algebra

%\newcommand{\tablesize}{\fontsize{9}{11}\selectfont} % tamanho tabela

\allowdisplaybreaks[4] % para evitar espaços entre fórmulas

\renewcommand{\titlesize}{\Huge}


%%comandos exclusivos deste poster.
%\hypersetup{pdfpagelayout=SinglePage} %abre a pagina em modo simples
\geometry{paperwidth=90cm,paperheight=125cm,centering,
    textwidth=80cm,textheight=120cm,left=2.2cm,top=1.5cm}
%********************************************************************
\begin{document}

%%titulo
\colorbox{BoxCol}{
  \begin{minipage}{\textwidth}
    \color{white}{
      \begin{center}
        \Huge{\textbf{\vspace{0.2cm} \\
        Modelo de regressão Quantílico com Base na Distribuição Dagum Exponencial Generalizada Exponencializada
        \vspace{0.7cm} \\}}
      \end{center}
    }
  \end{minipage}
}
\title{} % n\~ao delete
\author{%\hspace{3.5cm}
\textbf{Vítor Bernardo Silveira Pereira$^1$ \& Cleber Bisognin$^2$ \& Laís Helen Loose$^2$ } 
} %%<-- seu nome
\institute{%\hspace{3.5cm}
%\vspace{-0.4cm}
$^1$Graduando em Estatística, UFSM - \texttt{vitorpereira3115@gmail.com}\\
$^2$Departamento de Estatística, UFSM - \texttt{cleber.bisognin@ufsm.br \& lais.loose@ufsm.br}}\\
\vspace{-0.1cm}
%\rightlogo[1.2]{abe}\hspace{25cm} %logotipo da abe
\rightlogo[1.1]{brasao_cores}%\hspace{25cm} %logotipo da abe
\leftlogo[1.3]{logo} %logotipo da ufsm

\maketitle %gera titulo

\vspace{-2.7cm}
\section*{\hspace{2cm}\vspace{0.0cm}\tituloA{24º Simpósio Nacional de Probabilidade e Estatística - 31 de julho a 5 de agosto de 2022}}
%Texto em 2 colunas
\begin{multicols}{3}{

%Paragrafo.
\setlength{\parindent}{0.5em}

\section*{\tituloA{Motiva\c{c}\~ao}}
\vspace{0.2cm}

Para Ferrari e Cribari-Neto \cite{Ferrari2004} o uso de modelos de regressão destaca-se na análise de dados que são correlacionados com outras variáveis. O modelo de regressão normal é o mais conhecido e utilizado em análise de regressão onde pressupomos que os erros do modelo são normalmente distribuídos. 

Alternativamente, temos proposições como os modelos lineares generalizados (MLGs). Nestes modelos supomos que a variável de interesse possui distribuição que pertence a uma classe de distribuições que pertencem a família exponencial natural. Porém tanto no modelo de regressão normal quanto no MLG busca-se modelar a média da variável resposta. 

No entanto, os pressupostos desses modelos podem não ser cumpridos em inúmeros estudos, assim a utilização de distribuições clássicas para ajustar diferentes conjuntos de dados pode levar a ajustes imprecisos. Para flexilizar esses pressupostos, temos a proposição de novos modelos, como, a regressão quantílica que busca modelar um quantil da variável resposta e não a média. 

Além disso a regressão quantílica, de acordo com \cite{mazucheli2020} nos permite uma abordagem um pouco mais completa permitindo estudar a cauda da distribuição e seus extremos. O autor também afirma que se houver valores muito discrepantes na variável dependente condicional, a mediana pode se tornar mais apropriada que a média. 

Nesse contexto, o presente trabalho tem como objetivo propor um novo modelo de regressão quantílica para modelagem de quantis de distribuições contínuas e positivas utilizando a distribuição Dagum Exponencial Generalizada Exponencializada (EGED), a qual foi proposta por Suleman Nasiru \cite{nasiru2019exponentiated}, Peter N. Mwita e Oscar Ngesa em $2019$. 

A distribuição EGED é utilizada para modelagem de dados contínuos e positivos, com aplicações nas áreas de análise de confiabilidade, finanças (renda pessoal e distribuições riqueza), ciências atuariais entre outros de acordo com \textit{A guide to the Dagum Distributions} \cite{kleiber2008guide} .

\section*{\tituloA{Distribui\c{c}\~ao EGED e Reparametriza\c{c}\~ao Quantílica}}
\vspace{0.2cm}

Seja $Y$ uma variável aleatória com distribuição Dagum Exponencial Generalizada Exponencializada denotada por EGED $(\alpha,\sigma,\delta,\lambda,\eta,\gamma)$. 

Sua função densidade de probabilidade é dada pela equação \eqref{eq1}:

\begin{align}\label{eq1}
f_Y(y)= \frac{\alpha\sigma\lambda\delta\eta\gamma(1+\alpha y^{-\delta})^{-\sigma-1}[1-g(y)]^{\gamma-1}[1-(1-g(y)^{\gamma}]^{\eta-1}}{y^{\delta+1}{1-[1-(1-g(y)^{\gamma}]^{\eta}\}^{1-\lambda}}}
\end{align}
com $y >0$, onde $g(y) = (1+\alpha y^{-\delta})^{-\sigma}$ onde os parâmetros $\alpha,\sigma,\delta,\lambda,\eta,\gamma$ são não negativos, assim temos que  $\sigma,\delta,\lambda,\eta$ e $\gamma$  são os parâmetros de forma e apenas $\alpha$ como o parâmetro de escala. 

A função de distribuição acumulada de $Y$ é dada pela equação \eqref{eq2} abaixo

\begin{equation}\label{eq2}
F_Y(y)=1-\left\{1-\left[1-\left(1-\left(1+\alpha y^{-\delta}\right)^{-\sigma}\right)^{\gamma}\right]^{\eta}\right\}^{\lambda}, \quad  y > 0.
\end{equation}

A função quantílica é dada por:

\begin{equation}\label{eq3}
Q_Y(\tau)=\left\{\frac{1}{\alpha}\left[\left(1-\left(1-\left(1-\tau^\frac{1}{\lambda}\right)^\frac{1}{c}\right)^{\frac{1}{d}}\right)^{-\frac{1}{\sigma}}-1\right]\right\}^{-\frac{1}{\delta}}, 
\end{equation}
onde $\tau \in (0,1)$, a fim de obter o modelo de regressão quantílica reparametrizamos as equações \eqref{eq1} e \eqref{eq2} em termos do quantil. Para isso utilizaremos a relação $\mu = Q_Y(\tau)$. Pode-se utilizar a reparametrização em termos de $\delta$ ou $\alpha$, dadas abaixo:

\begin{equation}\label{eq4}
\delta=-\frac{1}{\log(\mu_t)}\log\left\{\frac{1}{\alpha}\left[\left(1-\left(1-\left(1-\left(1-\tau\right)^\frac{1}{\lambda}\right)^\frac{1}{\eta}\right)^{\frac{1}{\gamma}}\right)^{-\frac{1}{\sigma}}-1\right]\right\},
\end{equation}


\begin{equation}\label{eq5}
\alpha=\exp\left[\delta\log(\mu_t)\right]\left[\left(1-\left(1-\left(1-\left(1-\tau\right)^\frac{1}{\lambda}\right)^\frac{1}{\eta}\right)^{\frac{1}{\gamma}}\right)^{-\frac{1}{\sigma}}-1\right]
\end{equation}

Logo, a função densidade de probabilidade da distribuição EGED, dada pela equação \eqref{eq1}, pode ser reparametrizada em termos do quantil ($\tau$), substituindo a equação \eqref{eq4} ou \eqref{eq5} na função \eqref{eq1}.
\vspace{4cm}

\section*{\tituloA{Modelo de Regress\~ao Quantílica EGED}}
\vspace{0.2cm}

O modelo de regressão quantílica EGED é obtido assumindo que cada ${\mu_t}$ pode ser escrito na estrutura de regressão dada abaixo: 
\begin{equation}\label{eq6}
 g(\mu_t)=\bm{x}_t^{\top}\bm{\beta}=\eta_t, \quad t=1, \ldots, n,
\end{equation}

em que $\bm{\beta}=(\beta_0,\beta_1, \ldots, \beta_k)^{\top}$  é o vetor de parâmetros desconhecidos ($\bm{\beta}\in\mathbb{R}^{k+1}$) e $\bm{x}_{t}=({x_{t0}},\cdots,{x_{tk+1}})^\top$ são observações de $k+1$ covariáveis ($k+1<n$), as quais são supostamente fixas e conhecidas. 

A função $g(\cdot)$ é chamada de função de ligação e é estritamente monótona e duas vezes diferenciável, tal que $g:\mathbb{R}^{+}\rightarrow \mathbb{R}$. Devido à restrição $\mu_t > 0$ a função de ligação mais usual é a função logarítmica, $g(\mu_t)=\log(\mu_t)$.

Considere $\bm{y}=(y_1,...,y_n)$ uma amostra aleatória do modelo de regressão quantílico em \eqref{eq6}. Logo, a função de log-verossimilhança de $\bm{y}$ é dada por
 
\vspace{-0.5cm} $$\ell(\bm{\theta})=\ell(\bm{\theta};\bm{y})=\sum_{t=1}^{n}\ell_t(\alpha,\sigma,\delta,\lambda,\eta,\gamma),$$

\begin{eqnarray*}
\ell_t(\alpha,\sigma,\delta,\lambda,\eta,\gamma)&=&\log(\alpha\lambda\sigma\delta\eta\gamma)-(\delta+1)\log(y_t)-(\sigma+1)\log(z_t)\nonumber\\
\\
\vspace{-0.3cm}&&+(\gamma+1)\log\left(1-z_t^{-\sigma}\right)\nonumber\\
\\
\vspace{-0.3cm}&&+(\eta-1)\log\left[1-\left(1-z_t^{-\sigma}\right)^\gamma\right]\nonumber\\
\\
\vspace{-0.3cm}&&+(\lambda-1)\log\left\{1-\left[1-\left(1-z_t^{-\sigma}\right)^{\gamma}\right]^{\eta}\right\},\nonumber
\end{eqnarray*}

\noindent com $z_t=(1+\alpha x_t^{-\delta})$, para encontrar as estimativas dos parâmetros usaremos o estimador de máxima verossimilhança (EMV), considerando $\bm{\theta}=(\alpha,\sigma,\lambda,\eta,\gamma,\bm{\beta}^{\top})^{\top}$ ou $\bm{\theta}=(\sigma,\delta,\lambda,\eta,\gamma,\bm{\beta}^{\top})^{\top}$ o vetor de parâmetros, utilizando a função de verossimilhança reparametrizando a função densidade de probabilidade dada pela equação \eqref{eq1}, pelas \eqref{eq4} ou \eqref{eq5}.

O EMV é tal que $\widehat{\boldsymbol{\theta}}=\mbox{argmax}_{{\boldsymbol{\theta}}\in \Theta}\{\ell(\boldsymbol{\theta})\}$, sendo obtido através da solução do sistema de equações não lineares dadas por:
\begin{align}\label{eq8}
\left\{ \begin{array}{l}
U_{\alpha}(\bm\theta) = 0\mbox{ ou }U_{\delta}(\bm\theta) = 0,\\
U_{\sigma}(\bm\theta) = 0,\\
U_{\lambda}(\bm\theta) = 0, \\
U_{\eta}(\bm\theta) = 0, \\
U_{\gamma}(\bm\theta) = 0, \\
\bm{U}_{\bm{\beta}}({\bm\theta}) = \bm{0}.
\end{array} \right.
\end{align}

Como esse sistema de equações não apresenta solução analítica se faz necessário o uso de algoritmos de otimização não lineares. Neste caso utilizados o método de Gradientes Conjugados (CG), porém outros métodos como Nelder-Mead também podem ser utilizados. A matriz de informação de Fisher não tem uma expressão em forma fechada, assim utilizamos a matriz de Informação observada.


\section*{\tituloA{Avaliação numérica dos estimadores}}
\vspace{0.2cm}
Para a avaliação numérica dos EMV um estudo de simulação foi realizado via simulações de Monte Carlo. A implementação computacional foi desenvolvida na linguagem \cite{R2022}. 

Foram utilizadas $1.000$ replicações, três tamanhos amostrais $n \in \{100, 200, 500\}$ e diferentes quantis, $\tau \in \{0.20; 0.50; 0.90\}$ e fixando os parâmetros como $\bm{\theta} = (\alpha, \lambda,\sigma,\eta,\gamma,\beta_0,\beta_1,\beta_2)=(3.5; 11.2; 3.0; 5.0; 3.0; 0.5; 0.2; 1.5)$ ou $\bm{\theta} = (\delta, \lambda,\sigma,\eta,\gamma,\beta_0,\beta_1,\beta_2)=(3.5; 11.2; 3.0; 5.0; 3.0; 0.5; 0.2; 1.5)$. 

A maximização da função de log-verossimilhança, foi realizada pelo método CG, sendo necessário um chute inicial para a maximização. Para fornecer o chute inicial utilizamos a função {\tt gosolnp} (Rsolnp), a qual utiliza inicialização randômica com múltiplos pontos iniciais. As covariáveis foram geradas independentemente utilizando a distribuição uniforme padrão, $\mbox{U} (0, 1)$, e mantidas constantes durante todas as replicações de Monte Carlo. Apresentamos os resultados de $2$ cenários, o primeiro com $\delta$ ($\alpha$ reparametrizado),  $n =200$ e $\tau = 0.2$, o segundo com $\alpha$ ($\delta$ reparametrizado) $n = 200$ e $\tau$= $0.5$.

As medidas de desempenho para avaliação do EMV utilizadas foram a média, o viés, desvio padrão (DP), o erro quadrático médio (EQM), coeficientes de assimetria (CA) e curtose (CC).


%==============================
%alpha= delta= 3.5 lambda= 11.2 sigma= 3 eta= 5 gama= 3 betas= 0.5 0.2 1.5 n= 200 method= CG q= 1 tau= 0.2
\textbf{Tabela 1.} Resultados da avaliação numérica dos estimadores do modelo de EGED, quando $\bm{\theta} = (\delta, \lambda,\sigma,\eta,\gamma,\beta_0,\beta_1,\beta_2)=(3.5; 11.2; 3.0; 5.0; 3.0; 0.5; 0.2; 1.5)$, $\tau =0.50$ e $n=200$, utilizando o método CG e reparametrização por $\alpha$ - Cenário $1$, via  simulação de Monte Carlo.

\begin{table}[H]

\centering
\begin{tabular}[t]{c|c|c|c|c|c|c}
\hline
& Média & DP & Viés & EQM & CA & CC\\
\hline
\multicolumn{7}{c}{Tamanho amostral = $50$} \\
\hline
$\delta$ & 6.524 & 2.939 & 3.024 & 17.786 & 0.036 & 1.37\\
$\lambda$ & 8.877 & 2.275 & -2.323 & 10.573 & -2.098 & 6.200\\
$\sigma$ & 4.091 & 3.319 & 1.091 & 12.204 & 0.689 & 1.983\\
$\eta$ & 5.408 & 2.460 & 0.408 & 6.219 & 0.089 & 2.377\\
$\gamma$ & 4.967 & 3.680 & 1.967 & 17.412 & 0.317 & 1.465\\
$\beta_0$ & 0.502 & 0.031 & 0.002 & 0.001 & 0.016 & 3.099\\
$\beta_1$ & 0.203 & 0.039 & 0.003 & 0.002 & 0.085 & 2.888\\
$\beta_2$ & 1.495 & 0.042 & -0.005 & 0.002 & -0.044 & 2.889\\
\hline
\multicolumn{7}{c}{Tamanho amostral = $200$} \\
\hline
$\delta$ & 4.824 & 2.223 & 1.324 & 6.695 & 1.343 & 3.684\\
$\lambda$ & 9.203 & 1.736 & -1.997 & 6.999 & -2.510 & 8.744\\
$\sigma$ & 3.283 & 2.332 & 0.283 & 5.517 & 1.304 & 4.381\\
$\eta$ & 5.216 & 1.391 & 0.216 & 1.982 & 0.943 & 4.965\\
$\gamma$ & 3.930 & 2.668 & 0.930 & 7.984 & 1.110 & 3.299\\
$\beta_0$ & 0.500 & 0.015 & 0.001 & 0.001 & 0.092 & 3.033\\
$\beta_1$ & 0.202 & 0.019 & 0.002 & 0.001 & -0.022 & 2.824\\
$\beta_2$ & 1.499 & 0.019 & -0.001 & 0.001 & -0.0434 & 3.147\\
\hline
\multicolumn{7}{c}{Tamanho amostral = $500$} \\
\hline
$\delta$ & 4.163 & 1.711 & 0.663 & 3.367 & 2.444 & 8.523\\
$\lambda$ & 9.499 & 1.347 & -1.700 & 4.706 & -3.274 & 13.587\\
$\sigma$ & 3.209 & 1.891 & 0.209 & 3.621 & 1.861 & 7.267\\
$\eta$ & 5.141 & 1.014 & 0.141 & 1.049 & 1.754 & 8.145\\
$\gamma$ & 3.740 & 2.033 & 0.740 & 4.685 & 1.563 & 5.612\\
$\beta_0$ & 0.500 & 0.009 & 0.001 & 0.001 & -0.001 & 2.788\\
$\beta_1$ & 0.200 & 0.013 & 0.001 & 0.001 & -0.040 & 2.928\\
$\beta_2$ & 1.499 & 0.011 & -0.001 & 0.001 & 0.115 & 3.060\\
\hline
\end{tabular}
\end{table}


%==============================
%alpha= delta= 3.5 lambda= 11.2 sigma= 3 eta= 5 gama= 3 betas= 0.5 0.2 1.5 n= 200 method= CG q= 1 tau= 0.5
\textbf{Tabela 2.} Resultados da avaliação numérica dos estimadores do modelo de Regressão EGED, quando $\bm{\theta} = (\alpha, \lambda,\sigma,\eta,\gamma,\beta_0,\beta_1,\beta_2)=(3.5; 11.2; 3.0; 5.0; 3.0; 0.5; 0.2; 1.5)$, $\tau =0.50$ e $n=200$, utilizando o método CG e reparametrização por $\delta$ - Cenário $2$, via simulação de Monte Carlo.

\begin{table}[H]

\centering
\begin{tabular}[t]{c|c|c|c|c|c|c}
\hline
& Média & DP & Viés & EQM & CA & CC\\
\hline
\multicolumn{7}{c}{Tamanho amostral = $50$} \\
\hline
$\delta$ & 6.674 & 3.106 & 3.174 & 19.726 & -0.281 & 1.583\\
$\lambda$ & 9.115 & 2.030 & -2.085 & 8.469 & -2.482 & 8.104\\
$\sigma$ & 5.373 & 2.423 & 2.373 & 11.503 & 0.652 & 2.064\\
$\eta$ & 3.874 & 2.438 & -1.126 & 7.211 & 0.356 & 2.366\\
$\gamma$ & 6.070 & 3.861 & 3.070 & 24.330 & -0.188 & 1.308\\
$\beta_0$ & 0.485 & 0.055 & -0.015 & 0.003 & 0.157 & 2.972\\
$\beta_1$ & 0.257 & 0.099 & 0.057 & 0.013 & 0.032 & 2.508\\
$\beta_2$ & 1.501 & 0.103 & 0.001 & 0.011 & -0.096 & 3.103\\
\hline
\multicolumn{7}{c}{Tamanho amostral = $200$} \\
\hline
$\delta$ & 4.986 & 2.404 & 1.486 & 7.986 & 0.894 & 2.730\\
$\lambda$ & 9.555 & 1.211 & -1.645 & 4.172 & -3.681 & 16.572\\
$\sigma$ & 3.521 & 1.187 & 0.521 & 1.680 & 2.822 & 11.690\\
$\eta$ & 4.451 & 1.365 & -0.549 & 2.164 & -0.838 & 4.689\\
$\gamma$ & 4.898 & 3.128 & 1.898 & 13.387 & 0.535 & 1.827\\
$\beta_0$ & 0.489 & 0.029 & -0.011 & 0.001 & -0.160 & 3.120\\
$\beta_1$ & 0.232 & 0.051 & 0.032 & 0.004 & 0.263 & 3.137\\
$\beta_2$ & 1.501 & 0.050 & 0.001 & 0.002 & -0.097 & 2.979\\
\hline
\multicolumn{7}{c}{Tamanho amostral = $500$} \\
\hline
$\delta$ & 4.530 & 1.579 & 1.030 & 3.554 & 1.396 & 5.136\\
$\lambda$ & 9.833 & 0.606 & -1.367 & 2.234 & -6.451 & 48.423\\
$\sigma$ & 3.139 & 0.513 & 0.139 & 0.283 & 5.816 & 45.469\\
$\eta$ & 4.585 & 0.799 & -0.415 & 0.812 & -1.811 & 8.184\\
$\gamma$ & 4.853 & 2.597 & 1.853 & 10.180 & 0.707 & 2.338\\
$\beta_0$ & 0.499 & 0.016 & -0.001 & 0.000 & -0.018 & 3.019\\
$\beta_1$ & 0.207 & 0.025 & 0.007 & 0.001 & -0.138 & 3.185\\
$\beta_2$ & 1.505 & 0.032 & 0.005 & 0.001 & 0.010 & 2.739\\
\hline
\end{tabular}
\end{table}

\vspace{0.5cm}

\textbf{Figura 1.} Gráficos de convergência para todos os parâmetros do Cenário $1$, em todos os quantis analisados, $\tau \in \{0.20; 0.50; 0.90\}$}

\begin{figure}[h!]
\begin{center}
\subfigure[Convergência de $\delta$]{\includegraphics[width=11.1cm]{figcdelta.png}}
\subfigure[Convergência de $\lambda$]{\includegraphics[width=11.1cm]{figclambda.png}}
\subfigure[Convergência de  $\sigma$]{\includegraphics[width=11.1cm]{figcsigma.png}}
\subfigure[Convergência de $\eta$]{\includegraphics[width=11.1cm]{figceta.png}}
\subfigure[Convergência de $\gamma$]{\includegraphics[width=11.1cm]{figcgama.png}}
\subfigure[Convergência de $\beta_1$]{\includegraphics[width=11.1cm]{figcbeta0.png}}
\subfigure[Convergência de $\beta_1$]{\includegraphics[width=11.1cm]{figcbeta1.png}}
\subfigure[Convergência de $\beta_2$]{\includegraphics[width=11.1cm]{figcbeta2.png}}
\end{center}
\end{figure}


\section*{\tituloA{Conclus\~ao}}
\vspace{0.2cm}

As estimativas, em especial para a estrutura de regressão, indicam bons desempenhos, com o viés se aproximando de $0$, com resultados numéricos semelhantes para ambas reparametrizações por $\alpha$ ou $\delta$. Quanto aos quantis percebe-se resultados mais acurados com a mediana, no entanto em todos os casos quando o tamanho amostral aumenta o viés, desvio padrão e o EQM tendem a $0$, indicando que os estimadores dos parâmetros do modelo são assintoticamente não viesados e consistentes.

Em trabalhos futuros serão investigados outros cenários, cenários com aplicações a dados reais e desenvolvimento de técnicas computacionais para a difusão do uso da EGED.


\section*{\tituloA{Agradecimentos}}

Os autores agradecem à FAPERGS e à Sigma Jr. Consultoria Estatística pelo auxílio financeiro.


\section*{\tituloA{Refer\^encias}}
\begingroup
\renewcommand{\section}[2]{}
\begin{thebibliography}{99}
\small

\bibitem{Ferrari2004} Ferrari, S. L. P.; Cribari-Neto, F. (2004). \textbf {Beta regression for modeling rates and proportions}. Journal of Applied Statistics, 31, 7.

%\bibitem{bourguignon2021regression} Bourguignon, Marcelo and do Nascimento, Fernando Ferraz (2021). \textbf{Regression models for exceedance data: a new approach}. Statistical Methods \& Applications, 30, 157-273.

\bibitem{mazucheli2020} Mazucheli, J. and Menezes, A. F. B. and Fernandes, L. B. and De Oliveira, R. P. and Ghitany, M. E. (2020). \textbf{The unit-Weibull distribution as an alternative to the Kumaraswamy distribution for the modeling of quantiles conditional on covariates}. Journal of Applied Statistics, 46, 954-974.

\bibitem{nasiru2019exponentiated} Nasiru, Suleman and Mwita, Peter N and Ngesa, Oscar (2019). \textbf{Exponentiated generalized exponential Dagum distribution}. Journal of King Saud University-Science, 31, 362-371.

\bibitem{kleiber2008guide} Kleiber, Christian (2008) \textbf{Modeling income distributions and Lorenz curves}. Springer, 2008, 97-117.


\bibitem{R2022} R Development Core Team, (2022). \textbf {R: A Language and Environment for Statistical Computing}. R Foundation for Statistical Computing, Vienna, Austria, ISBN 3-900051-07-0.



 
\end{thebibliography}

\endgroup


\end{multicols}


\end{document} 
